\subsection*{Solution 8}

\begin{itemize}
\item[(a)]

The iteration sequence
\[
z_{n+1} = 15z_n^2 + 3z_n + \frac{1}{16}
\]
is conjugate to the iteration sequence\hbm{D3}{2.1}
\[
w_{n+1} = w_n+d
\]
where
\[
d = \frac{15}{16} + \frac{3}{2} - \frac{9}{4} = \frac{15+24-36}{16} = \frac{3}{16}
\]
so, $w_{n+1} = w_n+\frac{3}{16}$.
The conjugating function is
\[
h(z) = 15z+\frac{1}{2}\times3 = 15z+\frac{3}{2}
\]
So, $w_0 = h(z_0) = h(0) = 0+\frac{3}{2} = \frac{3}{2}$

\item[(b)][FY,LK]

$P_{\frac{3}{16}}$ has fixed points at $z$, where $z^2+\frac{3}{16}=z$,
these are the solutions to the equation
\[
z^2-z+\frac{3}{16} = 0
\]
So
\[
z = \frac{ 1 \pm \sqrt{1 - 12/16} }{ 2 } = \frac{ 1 \pm \sqrt{1/4} }{ 2 } = \frac{1}{2}\pm\frac{1}{4}
\]
Hence, the fixed points of $P_{\frac{3}{16}}$ are $\frac{3}{4}$ and $\frac{1}{4}$.

Now, $P'_{\frac{3}{16}}(z) = 2z$, so
\[ \left|P'_{\frac{3}{16}}\left(\frac{3}{4}\right)\right| = \frac{6}{4} = \frac{3}{2} > 1 \]
and
\[ \left|P'_{\frac{3}{16}}\left(\frac{1}{4}\right)\right| = \frac{2}{4} = \frac{1}{2} < 1 \]
Hence, $\frac{1}{4}$ is an attracting fixed point and $\frac{3}{4}$
is a repelling one.\hbm{D3}{1.5}

\item[(b)][VC]

Alternatively, to solve the quadratic equation, we can multiply both
sides by 16, so
\begin{eqnarray*}
&&
	z^2 - z + \frac{3}{16} = 0 \\
&\Leftrightarrow&
	16z^2 - 16z + 3 = 0 \\
&\Leftrightarrow&
	(4z - 1)(4z - 3) = 0
\end{eqnarray*}
Hence, the roots are $\frac{1}{4}$ and $\frac{3}{4}$.\footnote{and no
pesky formula in sight!}

\item[(c)]

Let $c=-\frac{3}{2}+i$, then it appears from the diagram that $c$ is
outside the Mandelbrot set.\hbm{D3}{4.3}

Using the specification for $M$\hbm{D3}{4.5}
\[
|P_c(0)| = |-3/2+i| = \sqrt{9/4+1} = \sqrt{13/4} < 2
\]
We go for the next iteration:
\begin{eqnarray*}
|P_c^{(2)}(0)|
	&=& |(-3/2+i)^2-3/2+i| \\
	&=& |9/4-1-3i-3/2+i| \\
	&=& |-1/4-2i| \\
	&=& \sqrt{1/16+4} \\
	&=& \sqrt{65/4} \\
	&\simeq& 4.0 > 2
\end{eqnarray*}
Hence, $c$ lies outside the Mandelbrot set, $c\not\in M$.

\end{itemize}

