\subsection*{Solution 6}

\begin{itemize}
\item[(a)]

We apply Rouch\'{e}'s Theorem for both cases.\hbm{C2}{2.4}

\begin{itemize}
\item[(i)]

Let $g_1(z)=iz^5$, then
\[
|f(z)-g_1(z)| = |5z^2-3i| \le 5|z|^2 +|3i| = 23 < 32 = |g_1(z)|
\]
Since, $f$ and $g_1$ are analytic on $\Complex$ and $C_1$ is a
simple-closed contour in $\Complex$, $f$ has the same number of zeros
as $g_1$ inside $C_1$, namely 5, and none on $C_1$.

\item[(ii)]

Let $g_2(z)=5z^2$, then
\[
|f(z)-g_2(z)| = |iz^5-3i| \le |z|^5+|-3i| = 4 < 5 = |g_2(z)|
\]
Since, $f$ and $g_2$ are analytic on $\Complex$ and $C_2$ is a
simple-closed contour in $\Complex$, $f$ has the same number of zeros
as $g_2$ inside $C_2$, namely 2, and none on $C_2$.

\end{itemize}

\item[(b)]

From part (a) we know that $f$ has $5-2=3$ zeros in the annulus
$\{z:1\le|z|<2\}$.
Now, since for $|z|=1$
\[
|f(z)| = |iz^5+5z^2-3i| \ge |iz^5| - 5|z|^2 - |3i| = 9 > 0
\]
then $f$ has no zeros on $C_2$, so it has exactly 3 zeros in the open
annulus $\{z:1<|z|<2\}$, hence it follows that $f(z)=0$ has 3 solutions
in the annulus.

\end{itemize}

