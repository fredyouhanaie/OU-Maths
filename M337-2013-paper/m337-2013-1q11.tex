\subsection*{Solution 11}

\begin{itemize}
\item[(a)]

We shall use the Cover-up rule\hbm{C1}{1.3} to obtain the residues at
the three simple poles at $0$, $\frac{3}{4}i$ and $-\frac{3}{4}i$:
\[
f(z)	= \frac{ \pi\cot(\pi z) }{ 16z^2+9) }
	= \frac{ \pi \cos(\pi z) }{ (4z-3i)(4z+3i)\sin(\pi z) }
\]
Hence,
\begin{eqnarray*}
\Res(f,0)
	&=& \frac{ \pi \cos(\pi z) }{ (4z-3i)(4z+3i) } \\
	&=& \frac{ \pi }{ (-3i)(3i) } \\
	&=& \frac{ \pi }{ 9 } \\
\\
\Res\left(f,\frac{3}{4}i\right)
	&=& \frac{ \pi \cos(\pi z) }{ 4(4z+3i)\sin(\pi z) } \\
	&=& \frac{ \pi/\sqrt{2} }{ 4(3i+3i)/(-\sqrt{2}) } \\
	&=& -\frac{ \pi }{ 24i } \\
	&=& \frac{\pi}{24}i \\
\\
\Res\left(f,-\frac{3}{4}i\right)
	&=& \frac{ \pi \cos(\pi z) }{ 4(4z-3i)\sin(\pi z) } \\
	&=& \frac{ -\pi/\sqrt{2} }{ 4(-3i-3i)/(-\sqrt{2}) } \\
	&=& \frac{ \pi }{ 24i } \\
	&=& -\frac{\pi}{24}i
\end{eqnarray*}

\item[(b)]
\todo
\item[(c)]
\todo
\end{itemize}

