%
%	This is the original solution for Q11 supplied by Dominic Corbett
%
\documentclass[10pt,a4paper]{article}
\usepackage{amsmath}
\usepackage{amsfonts}
\usepackage{amssymb}
\newcommand{\deriv}[2]{\frac{\mathrm{d}#1}{\mathrm{d}#2}}
\newcommand{\nicederiv}[2]{\nicefrac{\mathrm{d}#1}{\mathrm{d}#2}}
\newcommand{\hderiv}[3]{\frac{\mathrm{d}^#1 #2}{\mathrm{d}#3^#1}}	% higher derivatives
\newcommand{\nicehderiv}[3]{\nicefrac{\mathrm{d}^#1 #2}{\mathrm{d}#3^#1}}
\newcommand{\pderiv}[2]{\frac{\partial #1}{\partial #2}}
\newcommand{\nicepderiv}[2]{\nicefrac{\partial #1}{\partial #2}}
\newcommand{\psderiv}[3]{\frac{\partial ^2#1}{\partial #2\partial #3}}
\newcommand{\phderiv}[3]{\frac{\partial^#1 #2}{\partial#3^#1}}	% higher pure partial derivatives
\newcommand{\nicepsderiv}[3]{\nicefrac{\partial ^2#1}{\partial #2\partial #3}}
\newcommand{\dd}{\, \mathrm{d}}
\newcommand{\e}{\mathrm{e}}
\newcommand{\ii}{\mathrm{i}}
\newcommand{\N}{\mathbb{N}}
\newcommand{\Z}{\mathbb{Z}}
\newcommand{\R}{\mathbb{R}}
\newcommand{\Q}{\mathbb{Q}}
\newcommand{\C}{\mathbb{C}}
\newcommand{\Rr}{\mathcal{R}}
\newcommand{\Ss}{\mathcal{S}}
\newcommand{\Tt}{\mathcal{T}}
\newcommand{\ve}[1]{\mathbf{#1}}
\newcommand{\st}{\ensuremath{^\mathrm{st}}}
\newcommand{\nd}{\ensuremath{^\mathrm{nd}}}
\newcommand{\rd}{\ensuremath{^\mathrm{rd}}}
\newcommand{\nth}{\ensuremath{^\mathrm{th}}}
\newcommand{\comb}[2]{^{#1}C_{#2}}
\newcommand{\perm}[2]{^{#1}P_{#2}}
\DeclareMathOperator{\re}{Re}
\DeclareMathOperator{\im}{Im}
\DeclareMathOperator{\Log}{Log}
\DeclareMathOperator{\Arg}{Arg}
\DeclareMathOperator{\Wnd}{Wnd}
\DeclareMathOperator{\Res}{Res}
\DeclareMathOperator{\Ker}{Ker}
\DeclareMathOperator{\Orb}{Orb}
\DeclareMathOperator{\Stab}{Stab}
\DeclareMathOperator{\Fix}{Fix}
\DeclareMathOperator{\cosec}{cosec}
\DeclareMathOperator{\sech}{sech}
\DeclareMathOperator{\cosech}{cosech}
\newcommand{\gv}[1]{\mbox{\boldmath$ #1 $}}						% for vectors of Greek letters
\newcommand{\uv}[1]{\mathbf{\hat{#1}}}							% for unit vector
\newcommand{\grad}[1]{\gv{\nabla} #1}								% for gradient
\let\divsymb=\div												% rename builtin command \div to \divsymb
\renewcommand{\div}[1]{\gv{\nabla} \cdot #1}						% for divergence
\newcommand{\curl}[1]{\gv{\nabla} \times #1}						% for curl
\newcommand\Fr{\mathit{Fr}}										% for the Froude number
\newcommand\Reynolds{\mathit{Re}}								% for the Reynolds number
%% column vectors section ->
\newcount\colveccount
\newcommand*\colvec[1]{
        \global\colveccount#1
        \begin{pmatrix}
        \colvecnext
}
\def\colvecnext#1{
        #1
        \global\advance\colveccount-1
        \ifnum\colveccount>0
                \\
                \expandafter\colvecnext
        \else
                \end{pmatrix}
        \fi
}
%% <- column vectors section
\usepackage{fullpage}
\usepackage{tensor}
\delimitershortfall-1sp
\newcommand{\del}[1]{\left(#1\right)}
\newcommand{\cbr}[1]{\left\{#1\right\}}
\newcommand{\sbr}[1]{\left[#1\right]}
\newcommand{\abs}[1]{\left\lvert#1\right\rvert}
\newcommand{\norm}[1]{\left\lVert#1\right\rVert}
\newcommand{\intcc}[2]{\left[#1 , #2\right]}
\newcommand{\intoc}[2]{\left]#1 , #2\right]}
\newcommand{\intco}[2]{\left[#1 , #2\right[}
\newcommand{\intoo}[2]{\left]#1 , #2\right[}
\newcommand{\eval}[2]{\left. #1 \right|_{ #2 }}
\usepackage{bigints}
\usepackage{lastpage}
\usepackage{graphicx}
\usepackage{tikz}
\usepackage{caption}
\usepackage{subcaption}
\usepackage{grffile}
\usepackage{siunitx}
\usepackage{nicefrac}
\usepackage[normalem]{ulem}
%% left margin-aligned headings ->
\usepackage[explicit]{titlesec}
\usepackage{enumitem}
\titleformat{\subsubsection}%
            [hang]% shape
            {}% format
            {\llap{% label
               Q\textsuperscript{n}\thesubsubsection\hskip 9pt}#1}%
            {0pt}% horizontal sep
            {}% before

\titleformat{\paragraph}%
        [runin]% shape
        {}% format
        {\llap{% label
           \theparagraph\hskip 9pt}#1}%
        {0pt}% horizontal sep
        {}% before
        
\titleformat{\subparagraph}%
        [runin]% shape
        {}% format
        {\llap{% label
           \thesubparagraph\hskip 9pt}#1}%
        {0pt}% horizontal sep
        {}% before

\titlespacing*{\subparagraph}{0pt}{*1}{0em}
%% <- left margin-aligned headings
\usepackage{fancyhdr}
\pagestyle{fancy}
\lhead{M337 --- 2013 Exam Solutions}
\chead{}
\rhead{DBJC}
\headsep = 10pt
\lfoot{}
\cfoot{\thepage\ of \pageref{LastPage}}
\rfoot{}
\renewcommand{\headheight}{12.0pt}
\renewcommand{\headrulewidth}{0.4pt}
\renewcommand{\footrulewidth}{0.4pt}
\definecolor{light-gray}{gray}{0.75}
\definecolor{light-light-gray}{gray}{0.875}
\usetikzlibrary{arrows,decorations.markings,intersections}
\setcounter{secnumdepth}{5}
\def\thesubsubsection{\arabic{subsubsection}}
\def\theparagraph{(\alph{paragraph})}
\def\thesubparagraph{(\roman{subparagraph})}
\numberwithin{equation}{subsubsection}
\numberwithin{figure}{subsubsection}
\setcounter{subsubsection}{10}
\usepackage{xr-hyper}
\usepackage[bookmarks=true,backref=section,colorlinks=true,linkcolor=black,pdftitle={M337 2013 Exam Solutions},pdfauthor={DBJC}]{hyperref}
\begin{document}
\subsubsection{}
\paragraph{Note that}
\begin{align*}
	f ( z )
	&=
	\frac{ \pi \cos ( \pi z ) }{ \del{ 4 z + 3 i } \del{ 4 z - 3 i } \sin ( \pi z ) }
\intertext{so by the cover-up rule (\textit{HB28---1.3})}
	\Res ( f , - i \tfrac{3}{4} )
	&=
	\frac{ \pi \cos ( - \pi i \tfrac{3}{4} ) }{ 4 \del{ 4 ( - i \tfrac{3}{4} ) - 3 i } \sin ( - \pi i \tfrac{3}{4} ) }
	\\
	&=
	\frac{ \pi \cosh ( - \tfrac{3}{4} \pi ) }{ - 24 i^2 \sinh ( - \tfrac{3}{4} \pi ) }
	\\
	&=
	\underline{\underline{
	- \tfrac{ \pi }{ 24 } \coth ( \tfrac{3}{4} \pi )
	}}
\intertext{and}
	\Res ( f , i \tfrac{3}{4} )
	&=
	\frac{ \pi \cos ( \pi i \tfrac{3}{4} ) }{ 4 \del{ 4 ( i \tfrac{3}{4} ) + 3 i } \sin ( \pi i \tfrac{3}{4} ) }
	\\
	&=
	\frac{ \pi \cosh ( \tfrac{3}{4} \pi ) }{ 24 i^2 \sinh ( \tfrac{3}{4} \pi ) }
	\\
	&=
	\underline{\underline{
	- \tfrac{ \pi }{ 24 } \coth ( \tfrac{3}{4} \pi )
	}}
\intertext{and by the $ g/h $ rule (\textit{HB28---1.2})}
	\Res ( f , 0 )
	&=
	\frac{ \pi \cos ( \pi (0) ) }{ \del{ 4 (0) + 3 i } \del{ 4 (0) - 3 i } \pi \cos ( \pi (0) ) }
	\\
	&=
	\frac{ 1 }{ ( 3 i ) ( - 3 i ) }
	\\
	&=
	\underline{\underline{
	\tfrac{ 1 }{ 9 }
	}}
\end{align*}
\paragraph{$ \phi ( n ) = \frac{1}{16 n^2 + 9} $ is an even function which is analytic on $ \C $ except for poles at the points $ \tfrac{3}{4} i $ and $ - \tfrac{3}{4} i $. Now, if $ S_N $ is the square contour with vertices at $ (N + \tfrac{1}{2})(\pm 1 \pm i) $, then its length is $ L = 4 ( 2 N + 1 ) $ and}
\label{11b}
\begin{align*}
	\abs{\cot \pi z}
	&\leq
	2
	&&\text{for $ z \in S_N $ (\textit{HB30---4.2})}
\intertext{and since $ \abs{ z } \geq N + \frac{1}{2} $ for $ z \in S_N $, we have}
	\abs{ 16 z^2 + 9 }
	&\geq
	\abs{ \abs{ 16 z^2 } - \abs{ 9 } }
	&&(\textit{HB11---5.2})
	\\
	&=
	\abs{ 16 \abs{ z }^2 - 9 }
	\\
	&\geq
	\abs{ 16 ( N + \tfrac{1}{2} )^2 - 9 }
	&&\text{for $ z \in S_N $}
	\\
	\therefore
	\abs{ f ( z ) }
	=
	\abs{ \frac{ \pi \cot ( \pi z ) }{ 16 z^2 + 9 } }
	&\leq
	\frac{ 2 \pi }{ 16 ( N + \tfrac{1}{2} )^2 - 9 }
	=
	M
	&&\text{for $ z \in S_N $}
\intertext{Hence, by the Estimation Theorem,}
	\abs{ \int_{S_N} f ( z ) \dd z }
	&\leq
	\frac{ 2 \pi }{ 16 ( N + \tfrac{1}{2} )^2 - 9 } \cdot 4 ( 2 N + 1 ) ,
\intertext{which tends to $ 0 $ as $ N \to \infty $, that is}
	\lim_{N\to\infty} \int_{S_N} f ( z ) \dd z
	&=
	0
\intertext{It follows by \textit{HB30---4.1} that}
	\sum_{n=1}^\infty \frac{1}{16 n^2 + 9}
	&=
	- \tfrac{1}{2} \del{ \Res ( f , 0 ) + \Res ( f , - i \tfrac{3}{4} ) + \Res ( f , i \tfrac{3}{4} ) }
	\\
	&=
	- \tfrac{1}{2} \del{ \tfrac{ 1 }{ 9 } - \tfrac{ \pi }{ 24 } \coth ( \tfrac{3}{4} \pi ) - \tfrac{ \pi }{ 24 } \coth ( \tfrac{3}{4} \pi ) }
	\\
	&=
	\underline{\underline{
	\tfrac{ \pi }{ 24 } \coth ( \tfrac{3}{4} \pi ) - \tfrac{1}{18}
	}}
\end{align*}
\paragraph{Since}
\begin{align*}
	\sum_{n=-\infty}^\infty \frac{1}{16 n^2 + 9}
	&=
	\sum_{n=-\infty}^{-1} \frac{1}{16 n^2 + 9}
	+
	\frac{1}{16 (0)^2 + 9}
	+
	\sum_{n=1}^\infty \frac{1}{16 n^2 + 9}
	\\
	&=
	\sum_{n=\infty}^{1} \frac{1}{16 (-n)^2 + 9}
	+
	\frac{1}{9}
	+
	\sum_{n=1}^\infty \frac{1}{16 n^2 + 9}
	\\
	&=
	\frac{1}{9}
	+
	2 \sum_{n=1}^\infty \frac{1}{16 n^2 + 9}
	&&\text{(sum of positive reals indp't of order)}
\intertext{we can simply substitute in our result from part~\ref{11b} to give}
	\sum_{n=-\infty}^\infty \frac{1}{16 n^2 + 9}
	&=
	\frac{1}{9}
	+
	\tfrac{ 2 \pi }{ 24 } \coth ( \tfrac{3}{4} \pi ) - \tfrac{2}{18}
	\\
	&=
	\underline{\underline{
	\tfrac{ \pi }{ 12 } \coth ( \tfrac{3}{4} \pi )
	}}
	&&\textsc{qed}
\end{align*}

{\noindent}\rule{\textwidth}{1pt}
\end{document}
