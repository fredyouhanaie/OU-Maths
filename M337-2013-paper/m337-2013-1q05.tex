\subsection*{Solution 5}

\begin{itemize}
\item[(a)]

$f$ has three simple poles at 0, 1/5 and 5. We shall use the cover-up
rule to obtain the residues.\hbm{C1}{1.3}
%%
\begin{eqnarray*}
\Res(f, 0)
	&=& \frac{ z^2+1 }{ (5z-1)(z-5) } \\
	&=& \frac{1}{(-1)(-5)} \\
	&=& \frac{1}{5} \\
\\
\Res(f, 1/5)
	&=& \frac{ z^2+1 }{ 5z(z-5) } \\
	&=& \frac{ (1/5)^2+1 }{ 5(1/5)(1/5-5) } \\
	&=& \frac{ 26/25 }{ -24/5 } \\
	&=& -\frac{13}{60} \\
\\
\Res(f, 5)
	&=& \frac{ z^2+1 }{ z(5z-1) } \\
	&=& \frac{ 25+1 }{ 5(25-1) } \\
	&=& \frac{13}{60}
\end{eqnarray*}

\item[(b)]

We shall use the strategy for evaluating
$\int_0^{2\pi}\Phi(\cos t, \sin t)\,dt$.
\hbm{C1}{2.2}
After the replacements, we have, for $C=\{z:|z|=1\}$
\begin{eqnarray*}
\int_0^{2\pi} \frac{ \cos t }{ 13-5\cos t }\,dt
	&=& \int_C \frac{ \frac{1}{2}(z+1/z) }{ 13-\frac{5}{2}(z+1/z) } \times \frac{1}{iz}\,dz \\
	&=& \int_C \frac{ z^2+1 }{ 26z-5z^2-5 } \times \frac{1}{iz}\,dz \\
	&=& i \int_C \frac{ z^2+1 }{ z(5z^2-26z+5) }\,dz \\
	&=& i \int_C \frac{ z^2+1 }{ z(5z-1)(z-5) }\,dz \\
	&=& i \int_C f(z)\,dz
\end{eqnarray*}
%%
Now, $f(z)$ is analytic on $\Complex$, a simply-connected region,
except for the three singularities. The unit circle $C$ is a
simple-closed contour in $\Complex$, which does not pass through $f$'s
singularities, then by Cauchy's Residue Theorem\hbm{C1}{2.1}
\begin{eqnarray*}
\int_C f(z)\,dz
	&=& 2\pi i\left(\Res(f,0)+\Res(f,1/5)\right) \\
	&=& 2\pi i\left(\frac{1}{5}-\frac{13}{60}\right) \\
	&=& -\frac{\pi i}{30}
\end{eqnarray*}
%%
Hence,
\[
\int_0^{2\pi} \frac{ \cos t }{ 13-5\cos t }\,dt
	= i \int_C f(z)\,dz
	= i \left( -\frac{\pi i}{30} \right)
	= \frac{\pi}{30}
\]

\end{itemize}

